\documentclass[11pt]{book}
\usepackage[b5paper]{geometry}

% 겉표지, 속표지 등을 위한 종이 크기 재설정
% taken from
% http://faq.ktug.org/faq/LayOut/%c6%c7%c7%fc#s-4
%\special{pdf: pagesize width 190truemm height 260truemm}

%\usepackage[titles]{tocloft}
%\usepackage{titlesec}
\usepackage[doublespacing]{setspace} % double spacing

\usepackage{graphicx}
\graphicspath{{./figures/}} % figure seach path

\usepackage[utf]{kotex} % 한글 사용

% part numbering as arabic (or alph (a, b, ...), Alph (A, B, ...), roman (i, ii, ...), Roman (I, II, ...), fnsymbol (*, ....))
% taken from
% http://tex.stackexchange.com/questions/3177/how-to-change-the-numbering-of-part-chapter-section-to-alphabetical-r
%\renewcommand\thepart{\arabic{part}}

% change table of contents for part, i.e. Part I
% taken from
% http://tex.stackexchange.com/questions/38625/part-and-tocloft
\makeatletter
\def\@part[#1]#2{%
    \ifnum \c@secnumdepth >-2\relax
      \refstepcounter{part}%
      \addcontentsline{toc}{part}{\partname~\thepart\hspace{1em}#1}%
    \else
      \addcontentsline{toc}{part}{#1}%
    \fi
    \markboth{}{}%
    {\centering
     \interlinepenalty \@M
     \normalfont
     \ifnum \c@secnumdepth >-2\relax
       \huge\bfseries \partname\nobreakspace\thepart
       \par
       \vskip 20\p@
     \fi
     \Huge \bfseries #2\par}%
    \@endpart}
\makeatother

% Comment the following to have chapters numbered without interruption (numbering through parts)
% taken from
% http://tex.stackexchange.com/questions/35782/how-to-split-a-latex-document-using-parts-and-chapters
%\makeatletter\@addtoreset{chapter}{part}\makeatother

%\newlength\mylenchp
%\newlength\mylenapp
%
%\renewcommand\cftchappresnum{\chaptername~}
%\renewcommand\cftchapaftersnum{.}
%
%\settowidth\mylenchp{\cftchapfont\cftchappresnum\cftchapaftersnum}
%\settowidth\mylenapp{\cftchapfont\appendixname~\cftchapaftersnum}
%\addtolength\mylenchp{\cftchapnumwidth}
%\addtolength\mylenapp{\cftchapnumwidth}
%
%\setlength\cftchapnumwidth{\mylenchp}

\begin{document}

\tableofcontents

%\part{First Part}

\chapter{First Chapter}

\section{First Section}

\subsection{First Subsection}

\subsubsection{First Subsubsection}

Blah blah

\section{Second Section}

\subsection{Second Subsection}

\subsubsection{Second Subsubsection}

Blah blah

\section{Third Section}

\subsection{Third Subsection}

\subsubsection{Third Subsubsection}

Blah blah

\section{Fourth Section}

\subsection{Fourth Subsection}

\subsubsection{Fourth Subsubsection}

Blah blah

\chapter{Second Chapter}

\section{First Section}

\subsection{First Subsection}

\subsubsection{First Subsubsection}

Blah blah

\section{Second Section}

\subsection{Second Subsection}

\subsubsection{Second Subsubsection}

Blah blah

\section{Third Section}

\subsection{Third Subsection}

\subsubsection{Third Subsubsection}

Blah blah

\section{Fourth Section}

\subsection{Fourth Subsection}

\subsubsection{Fourth Subsubsection}

\chapter{Third Chapter}

\section{First Section}

\subsection{First Subsection}

\subsubsection{First Subsubsection}

Blah blah

\section{Second Section}

\subsection{Second Subsection}

\subsubsection{Second Subsubsection}

Blah blah

\section{Third Section}

\subsection{Third Subsection}

\subsubsection{Third Subsubsection}

Blah blah

\section{Fourth Section}

\subsection{Fourth Subsection}

\subsubsection{Fourth Subsubsection}

\chapter{Fourth Chapter}

\section{First Section}

\subsection{First Subsection}

\subsubsection{First Subsubsection}

Blah blah

\section{Second Section}

\subsection{Second Subsection}

\subsubsection{Second Subsubsection}

Blah blah

\section{Third Section}

\subsection{Third Subsection}

\subsubsection{Third Subsubsection}

Blah blah

\section{Fourth Section}

\subsection{Fourth Subsection}

\subsubsection{Fourth Subsubsection}

Reference example... \cite{Tsa2006}.

\addcontentsline{toc}{part}{References}
\bibliographystyle{IEEEtran}
\bibliography{bibliography}

\newpage

% footnote numbering to symbol
\renewcommand\thefootnote{\fnsymbol{footnote}}

% 한글 초록
이렇게... \footnote[1]{This is blah blah}

\end{document}