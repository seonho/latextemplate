\documentclass[11pt]{book}
\usepackage[b5paper]{geometry}

\usepackage[titletoc]{appendix}

% 겉표지, 속표지 등을 위한 종이 크기 재설정
% taken from
% http://faq.ktug.org/faq/LayOut/%c6%c7%c7%fc#s-4
%\special{pdf: pagesize width 190truemm height 260truemm}

%\usepackage[titles]{tocloft}
%\usepackage{titlesec}
\usepackage[doublespacing]{setspace} % double spacing

\usepackage{graphicx}
\graphicspath{{./figures/}} % figure seach path

\usepackage[utf]{kotex} % 한글 사용

% part numbering as arabic (or alph (a, b, ...), Alph (A, B, ...), roman (i, ii, ...), Roman (I, II, ...), fnsymbol (*, ....))
% taken from
% http://tex.stackexchange.com/questions/3177/how-to-change-the-numbering-of-part-chapter-section-to-alphabetical-r
%\renewcommand\thepart{\arabic{part}}

% change table of contents for part, i.e. Part I
% taken from
% http://tex.stackexchange.com/questions/38625/part-and-tocloft
\makeatletter
\def\@part[#1]#2{%
    \ifnum \c@secnumdepth >-2\relax
      \refstepcounter{part}%
      \addcontentsline{toc}{part}{\partname~\thepart\hspace{1em}#1}%
    \else
      \addcontentsline{toc}{part}{#1}%
    \fi
    \markboth{}{}%
    {\centering
     \interlinepenalty \@M
     \normalfont
     \ifnum \c@secnumdepth >-2\relax
       \huge\bfseries \partname\nobreakspace\thepart
       \par
       \vskip 20\p@
     \fi
     \Huge \bfseries #2\par}%
    \@endpart}
\makeatother

% Comment the following to have chapters numbered without interruption (numbering through parts)
% taken from
% http://tex.stackexchange.com/questions/35782/how-to-split-a-latex-document-using-parts-and-chapters
%\makeatletter\@addtoreset{chapter}{part}\makeatother

%\newlength\mylenchp
%\newlength\mylenapp
%
%\renewcommand\cftchappresnum{\chaptername~}
%\renewcommand\cftchapaftersnum{.}
%
%\settowidth\mylenchp{\cftchapfont\cftchappresnum\cftchapaftersnum}
%\settowidth\mylenapp{\cftchapfont\appendixname~\cftchapaftersnum}
%\addtolength\mylenchp{\cftchapnumwidth}
%\addtolength\mylenapp{\cftchapnumwidth}
%
%\setlength\cftchapnumwidth{\mylenchp}

\let\cleardoublepage\clearpage % blank page 제거

\newcommand{\hiddensection}[1]{
    \stepcounter{section}
    \section*{\Alph{chapter}.\arabic{section}\hspace{1em}{#1}}
}

\newcommand{\hiddensubsection}[1]{
    \stepcounter{subsection}
    \subsection*{\Alph{chapter}.\arabic{section}.\arabic{subsection}\hspace{1em}{#1}}
}

\usepackage{array}	%
\usepackage{multirow} % multicolumn or multirow
\usepackage{hhline}    % 

\usepackage{xcolor,colortbl} % for coloring 
\definecolor{lightgray}{gray}{0.9}

% Set default font for all the tables in this document
\let\oldtabular\tabular
\renewcommand{\tabular}{\footnotesize\oldtabular}

% Algorithm 관련 패키지 및 명령어 재정의
\usepackage{algorithm}  % algorithm package
\usepackage{algorithmic}

\renewcommand{\algorithmicrequire}{\textbf{Input:}}   % Algorithm input
\renewcommand{\algorithmicensure}{\textbf{Output:}} % Algorithm output
\renewcommand{\algorithmicforall}{\textbf{foreach}}    % Rename FORALL to foreach
\renewcommand{\algorithmicif}{\textbf{if}}                     % Rename IF to if
\renewcommand{\algorithmicendif}{\textbf{end}}          % Rename ENDIF to end
\renewcommand{\algorithmicendfor}{\textbf{end}}        % Rename ENDFOR to end

%\newcommand{\algname}[1]{{\sc #1}}                         % Set algorithm name

\usepackage{bm} % \bm command for bold italic math font

\usepackage{blindtext} % For long caption example

\usepackage{subfigure}

\begin{document}

\pagestyle{plain} % page 위치, header 제거

\tableofcontents
\listoffigures
\listoftables

\clearpage

\chapter{First Chapter}

\setcounter{page}{1}

\section{First Section}

\subsection{First Subsection}

\subsubsection{First Subsubsection}

Blah blah

In the following, Eqn.~\ref{eqn:cos2theta} illustrates double-angle formulae of $\cos$. % inline formula

% See http://en.wikibooks.org/wiki/LaTeX/Mathematics
% See http://en.wikibooks.org/wiki/LaTeX/Advanced_Mathematics
\begin{equation}
\label{eqn:cos2theta}
\cos (2\theta) = \cos^2 \theta - \sin^2 \theta
\end{equation}

\newpage

In the following, Fig.~\ref{fig:dum1} draws an example of figure.
% See http://en.wikibooks.org/wiki/LaTeX/Floats,_Figures_and_Captions
\begin{figure}[ht]
\begin{center}
\rule{0.5\linewidth}{0.35\linewidth}
\caption[Long caption figure]{\blindtext}
\label{fig:dum1}
\end{center}
\end{figure}

In the following, Fig.~\ref{fig:subfig} draws an example of subfigure.
\begin{figure}[ht]
\subfigure[This is a caption of the first subfig.]{
\includegraphics[width=2in]{subfig.eps}
}
\subfigure[This is a caption of the second subfig.]{
\includegraphics[width=2in]{subfig.eps}
}
\caption{This is a caption of the figure.}
\label{fig:subfig}
\end{figure}

In the following, Table~\ref{tbl:table1} denotes an example of table.
% See http://en.wikibooks.org/wiki/LaTeX/Tables for details
\begin{table}[ht]
\caption[Table example]{This is an example of table}
\label{tbl:table1}
\centering
\begin{tabular}{c c c}
\hline
A & B & C \tabularnewline \hline
\rowcolor{lightgray}
D & E & F \tabularnewline \hline
H & H & I \tabularnewline \hline
\end{tabular}
\end{table}

\begin{table}
\label{tbl:drandet}
\centering
\begin{tabular}{|c|c|c|c|c|c|c|c|}
\hline
\multirow{2}{*}{Samp. Lev.} & \multirow{2}{*}{EHMI Size} & \multicolumn{3}{c|}{Detection Rate (\%)} & \multicolumn{3}{c|}{Execution Time (ms)} \tabularnewline \hhline{~~------}
& & Rec.  & Prec. & F1 & Gen. EHMI & Com. HOG & Total\\
\hline
Level     8 & $  2687 \times 32$ & 0.22 & 100.00 & 0.44 & 5 & 1 & 10 \tabularnewline \hline
Level   18 & $  5882 \times 32$ & 5.41 & 93.59 & 10.23 & 7 & 1 & 12 \tabularnewline \hline
Level   28 & $  9426 \times 32$ & 51.45 & 93.15 & 66.28 & 8 & 2 & 14 \tabularnewline \hline
Level   38 & $13583 \times 32$ & 66.42 & 88.02 & 75.71 & 12 & 3 & 18 \tabularnewline \hline
Level   48 & $16790 \times 32$ & 75.24 & 87.88 & 81.07 & 13 & 4 & 21 \tabularnewline \hline
Level   58 & $20784 \times 32$ & 79.76 & 88.12 & 83.74 & 14 & 4 & 23 \tabularnewline \hline
\rowcolor{lightgray}
Level   68 & $24049 \times 32$ & 84.14 & 87.78 & 85.92 & 15 & 4 & 25 \tabularnewline \hline
Level   78 & $27363 \times 32$ & 84.58 & 87.77 & 86.15 & 18 & 5 & 29 \tabularnewline \hline
Level   88 & $29536 \times 32$ & 85.47 & 87.68 & 86.56 & 21 & 5 & 32 \tabularnewline \hline
Level   98 & $33031 \times 32$ & 86.66 & 88.03 & 87.34 & 24 & 5 & 36 \tabularnewline \hline
Level 108 & $35447 \times 32$ & 87.47 & 88.92 & 88.19 & 25 & 6 & 39 \tabularnewline \hline
\end{tabular}
\end{table}

\begin{table}
\label{tbl:tbl2}
\centering
\begin{tabular}{c|ccc}
\hline
& Recall  (\%) & Precision (\%) & F1-Measure (\%) \tabularnewline \hline
Part 0 & 4.0 & 66.7 & 7.5 \\
Part 12 & 46.0 & 92.0 & 61.3 \\
\rowcolor{lightgray}
Part 24 & 78.0 & 97.5 & 86.7 \\
Part 36 & 94.0 & 78.3 & 85.5 \\
Part 48 & 92.0 & 61.3 & 73.6 \\
Part 60 & 90.0 & 72.6 & 80.4 \\
\rowcolor{lightgray}
Part 72 & 88.0 & 93.6 & 90.7 \\
Part 84 & 2.0 & 50.0 & 3.8 \tabularnewline \hline
\end{tabular}
\end{table}

Alg.~\ref{alg:example} illustrates an example of algorithm.

% Algorithm - http://en.wikibooks.org/wiki/LaTeX/Algorithms
\begin{algorithm}[ht]
\caption{ALGORITHM EXAMPLE}
\label{alg:example}
\begin{algorithmic}[1]
\REQUIRE some inputs.
\ENSURE some outputs.

\STATE <text>
 \IF{<condition>} \STATE{<text>} \ENDIF
 \FOR{<condition>} \STATE{<text>} \ENDFOR
 \FOR{<condition> \TO <condition> } \STATE{<text>} \ENDFOR
 \FORALL{<condition>} \STATE{<text>} \ENDFOR
 \WHILE{<condition>} \STATE{<text>} \ENDWHILE
 \REPEAT \STATE{<text>} \UNTIL{<condition>}
 \LOOP \STATE{<text>} \ENDLOOP
 \RETURN <text>
 \PRINT <text>
 \COMMENT{<text>}
 \AND, \OR, \XOR, \NOT, \TO, \TRUE, \FALSE

\end{algorithmic}
\end{algorithm}

\section{Second Section}

\subsection{Second Subsection}

\subsubsection{Second Subsubsection}

\section{Third Section}

\subsection{Third Subsection}

\subsubsection{Third Subsubsection}

Blah blah

\section{Fourth Section}

\subsection{Fourth Subsection}

\subsubsection{Fourth Subsubsection}

Blah blah

\chapter{Second Chapter}

\section{First Section}

\subsection{First Subsection}

\subsubsection{First Subsubsection}

Blah blah

\section{Second Section}

\subsection{Second Subsection}

\subsubsection{Second Subsubsection}

Blah blah

\section{Third Section}

\subsection{Third Subsection}

\subsubsection{Third Subsubsection}

Blah blah

\section{Fourth Section}

\subsection{Fourth Subsection}

\subsubsection{Fourth Subsubsection}

\chapter{Third Chapter}

\section{First Section}

\subsection{First Subsection}

\subsubsection{First Subsubsection}

Blah blah

\section{Second Section}

\subsection{Second Subsection}

\subsubsection{Second Subsubsection}

Blah blah

\section{Third Section}

\subsection{Third Subsection}

\subsubsection{Third Subsubsection}

Blah blah

\section{Fourth Section}

\subsection{Fourth Subsection}

\subsubsection{Fourth Subsubsection}

\chapter{Fourth Chapter}

\section{First Section}

\subsection{First Subsection}

\subsubsection{First Subsubsection}

Blah blah

\section{Second Section}

\subsection{Second Subsection}

\subsubsection{Second Subsubsection}

Blah blah

\section{Third Section}

\subsection{Third Subsection}

\subsubsection{Third Subsubsection}

Blah blah

\section{Fourth Section}

\subsection{Fourth Subsection}

\subsubsection{Fourth Subsubsection}

Reference example... \cite{Tsa2006}.

\begin{appendices}

\chapter{First Appendix Chapter}

\hiddensection{First Appendix Section}
Blah blah

\hiddensubsection{First Subsection}
Blah blah

\subsubsection{First Subsubsection}
Blah blah

\hiddensection{Second Appendix Section}
Blah blah

\chapter{Second Appendix Chapter}

\hiddensection{First Appendix Section}
Blah blah

\hiddensection{Second Appendix Section}
Blah blah

\end{appendices}

\addcontentsline{toc}{part}{References}
\bibliographystyle{IEEEtran}
\bibliography{bibliography}

\addcontentsline{toc}{part}{Abstract}

\newpage

\pagestyle{plain}

% footnote numbering to symbol
\renewcommand\thefootnote{\fnsymbol{footnote}}

% 한글 초록
한글 초록은 여기에 이렇게... \footnote[1]{This is blah blah}

\end{document}